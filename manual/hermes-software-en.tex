\documentclass[11pt,a4paper]{article}
% \usepackage[brazil]{babel} % carrega portugues brasileiro
\usepackage[utf8]{inputenc}
\usepackage[T1]{fontenc}
\usepackage[top=2cm, bottom=2cm, left=2cm, right=2cm]{geometry} %margens menores!
\usepackage{graphicx} % incluir figuras .eps
\usepackage{tabularx}
\usepackage{color} % colorir texto
\usepackage{indentfirst}
\usepackage{textcomp}
\usepackage[colorlinks=true]{hyperref}
\usepackage{amssymb,amsmath}
\usepackage{float}
% \usepackage{siunitx}
% \usepackage[ampersand]{easylist}

\title{HERMES - High-frequency Emergency and Rural Multimedia Exchange System}

\author{
       \large
        \textsc{Rafael Diniz}
        \mbox{}\\ %
        rafael@rhizomatica.org\\
        \mbox{Rhizomatica} \\ %
        \normalsize
        \texttt{Brasília - Brasil}\\
}
\date{\today}


\begin{document}

\maketitle

\begin{abstract}

This document contains the description of the software stack of the HERMES
system - a digital communication system for the HF band which uses ARDOP
(Amateur Radio Digital Open Protocol) modem and UUCP networking (Unix to Unix
Communication Protocol). This document also contain a description of the
user interface and the minimum hardware needed.

\end{abstract}

\newpage

\tableofcontents

\section{Reference system}

\begin{itemize}
\item HF transceiver connected to a computer with audio I/O and ptt control
\item Mini computer running Linux (and the software stack described by this
  document) and connected to the transceiver and other networks
\end{itemize}

\section{What HERMES can do}

HERMES is as a system to be used with HF transceivers for digital
communication. HERMES is meant to work with HF NVIS (near vertical incidence
skywave) and provide hundreds of kilometers of reach for each station.

The modem we currently use (ARDOP) supports both point-to-point mode (ARQ
mode - Automatic repeat request) and broadcast mode (FEC mode - Forward
Error Correction). We only use the ARQ mode, which guarantees the error-free
reception of the data. Broadcast mode could be used, but using a protocol
different than UUCP.

We adopted UUCP as networking solution to be carried over the ARDOP modem in
ARQ mode. UUCP can carry files and execute remote commands. Specific support
for email exists and can be used out-of-the-box to provide email service over
HF (UUCP was created in the late 70's and one of the main uses was email).

In the more basic configuration, HERMES is used for file exchange, with the
option for secure (through cryptography) exchange of files.

\section{HERMES software configuration}

HERMES uses ARDOP (Amateur Radio Digital Open Protocol) for modem. ARDOP is
a SDR (software defined radio) modem made by amateur radio operators which
use modern modulation techniques (OFDM) and have support for standard and
homebrew transceivers.

For the network layer, UUCP is used. UUCP is a system for assynchronous
store-and-forward communication first released in the Bell Labs Unix V7 in
late 70's, and still used today in niches, like communication in HF.

To integrate the UUCP system with the ARDOP modem, the RHIZO-UUARDOPD
project was developed to provide tools that integrate UUCP to ARDOP.

A mini-computer running Linux is needed. We use Debian Buster (10) arm64
(multilib with armhf in case you want to use the `` piardopc '' by John
Wiseman) as Linux reference system, running on a Raspberry Pi with
Wifi configured in AP (Access Point) mode, for serving the system web
interface.

We split this section in the following topics:
\begin{itemize}
\item Radio / Interface
\item ARDOP
\item UUCP
\end{itemize}

\subsection{Radio and/or interface}

Different setups require different configuration. In the case of using a USB
(Universal Serial Bus) interface (eg. Signalink), the delay setting must be
set to 0. In the case of ICOM IC-7100, for example, leave the bandpass filter wider
than 2.8kHz for the digital mode  (SSB/Data).

\subsection{Alsa}

Add to ``/etc/asound.conf'':
\begin{verbatim}
pcm.ARDOP {type rate slave {pcm "hw:1,0" rate 48000}}
\end{verbatim}

\subsection{ARDOP}

Dowload link: \url{https://github.com/DigitalHERMES/ardopc}

The ardop binary used should be in /usr/bin/ardop, which can be a
symbolic link to /usr/bin/{ardop1ofdm, ardop2, ardopofdm}.

ARDOP service file for the ICOM IC-7100 (USB connection, PTT done over serial):
\begin{verbatim}
[Unit]
Description=ARDOP daemon

[Service]
Type=simple
ExecStart=/usr/bin/ardop 8515 -c /dev/ttyUSB0 ARDOP ARDOP -k FEFE88E01C0001FD -u FEFE88E01C0000FD
ExecStop=/usr/bin/killall -s QUIT ardop
IgnoreSIGPIPE=no
#StandardOutput=null
#StandardError=null
StandardOutput=syslog
StandardError=syslog

[Install]
WantedBy=multi-user.target
\end{verbatim}

Service file when using a VOX based setup (eg. when using an interface like
the Signalink):
\begin{verbatim}
[Unit]
Description=ARDOP daemon

[Service]
Type=simple
ExecStart=/usr/bin/ardop 8515 ARDOP ARDOP
ExecStop=/usr/bin/killall -s QUIT ardop
IgnoreSIGPIPE=no
#StandardOutput=null
#StandardError=null
StandardOutput=syslog
StandardError=syslog

[Install]
WantedBy=multi-user.target
\end{verbatim}


Start/stop ARDOP service:
\begin{verbatim}
systemctl start ardop.service
systemctl stop ardop.service
\end{verbatim}


See the log:
\begin{verbatim}
journalctl -f -u ardop
\end{verbatim}

\subsection{UUCP}

Use debian package version 1.07-27 or higher, for example, the version
from Debian Bullseye: \url{https://packages.debian.org/bullseye/uucp}
(eg. download and install .deb).

\subsection{rhizo-uuardopd}

%TODO: It seems everything uucico wrote in the end of a reception, don't get
%to the other side - connection closes and buffer get clean fast!

Two binaries should be installed: uuport (to be called by uucp) and uuardopd
which is the software that connects to the Ardop modem through uucico or uuport.

Download link: \url{http://github.com/DigitalHERMES/rhizo-uuardop}

Start/stop UUARDOPD service:
\begin{verbatim}
systemctl start uuardopd.service
systemctl stop uuardopd.service
\end{verbatim}


See the log:
\begin{verbatim}
journalctl -f -u uuardopd
\end{verbatim}

%\subsection{DHCP}

%\subsection{DNS}

%\subsection{Apache + PHP}

\section{UUCP command line examples}

We use ``/var/www/html/arquivos'' as default for received
files in the web interface, so the examples here use this path.

For copying file to a remote host. This adds a copy job to the uucp queue (``-r'' for not starting transmission
automatically):
\begin{verbatim}
uucp -C -r -d source.xxx AM4AAB\!/var/www/html/arquivos/${nodename}/
\end{verbatim}

To trigger the transmission of all queued jobs for host
AM4AAA:
\begin{verbatim}
uucico -S AM4AAA
\end{verbatim}

To list the jobs:
\begin{verbatim}
uustat -a
\end{verbatim}

To kill a job:
\begin{verbatim}
uustat -k job
uustat -K
\end{verbatim}

See the log:
\begin{verbatim}
uulog
\end{verbatim}

\section{User interface}

The system interface is web-based, using HTML + PHP and
some shell scripts.

Requirements of the user interface:

\begin{itemize}
\item ImageMagic: for image manipulation
\item mozjpeg: To compress for JPEG: https://github.com/mozilla/mozjpeg
% \item opusenc: para comprimir áudio
\item gpg: For cryptography
\item hostapd: Wifi AP mode software
\end{itemize}

The Web interface can be accessed by typing any address in a browser
connected to the wifi (set the DNS accordingly) or simply 192.168.1.1.

\subsection{Email}

Email server can either run locally or only in a central host with
Internet. Stations can either opt to connect to a central station on demand,
have some pre-defined schedule, or the central station connects to the
community stations doing a pooling, delivering and downloading emails and
files.

\subsection{WebPhone}

WIP!

\url{https://gitlab.tic-ac.org/keith/webphone/wikis/hermes}

\end{document}
